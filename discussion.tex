\section*{Discussion}

Our experimental approach allowed us to single out the effects of litter quality on the microbial decomposer community as well as decomposition processes, while excluding effects of fauna, climate and the initial microbial community. By exploiting intra-specific differences in beech litter stoichiometry, we were able to minimize differences in the chemical composition of initial litter (e.g. similar lignin and cellulose content, table \ref{initstoech}), while exploring the effect of litter nutrient contents on lignin and carbohydrate decomposition. Therefore, we can attribute different rates of carbohydrate and lignin decomposition to the intrinsic qualities of litter collected at different sites, i.e. elemental and stoichiometric composition.

Contradicting the traditional concepts of litter decomposition, our results demonstrate that relevant but variable amounts of lignin were degraded during the first 6 months of incubation. During this early stage, lignin decomposition rates depended on litter quality (P) and ranged from non-significant to degradation rates similar to bulk carbon mineralization rates (i.e. no discrimination against lignin). We can therefore confirm that early lignin decomposition rates are by far underestimated, as recently proposed by \cite{Klotzbucher2011}, based on a complementary analytic approach. Unlike them, we found no decreases but constant or increasing lignin decomposition rates during litter decomposition over 15 months. Additionally, we found a marked change in the controls of lignin decomposition during this period. While carbohydrate and lignin decomposition were differently controlled by litter chemistry (N versus P) during the first 6 months, these litter components were decomposed at similar rates thereafter and decomposition rates were only related to litter N availability.

Differences in initial lignin contents were marginal (29-31 \% relative peak area), and lignin contents of sites with high initial lignin decomposition rates were not higher than that of sites with low rates. Therefore, differences in early lignin decomposition did not result from high or low lignin contents as is suggested by traditional litter decomposition models. Low lignin decomposition rates were also not caused by a lack of Mn or Fe, the metals being important cofactors of oxidative lignin decay, which were suggested to be rate limiting during late lignin decomposition \cite{Berg2008}. While Mn and Fe concentrations strongly varied between litter collected at different sites, Mn and Fe concentrations were lowest in the litter with highest lignin decomposition rates (AK, see Table \ref{initstoech}). Low contents of these elements would explain decreased but not enhanced lignin decomposition. Moreover, soluble organic C was suggested to be limiting for lignin decomposition since the process of lignin decomposition does not generate enough energy for survival of lignin decomposers \cite{Klotzbucher2011}. Soluble organic C apparently did not control lignin decomposition since we found highest concentrations in the two litter types that showed the highest and the lowest lignin decomposition rates.

We found strong evidence that litter C:N:P stoichiometry and litter element concentrations exerted a major control on the extent of lignin decomposition during the initial decomposition phase. Carbohydrate decomposition was positively correlated with litter N contents and negatively to litter C:N ratios, as were the majority of decomposition processes (mass loss, respiration, potential extracellular enzymatic activities). In contrast, lignin decomposition rates were positively correlated with litter C:P ratios and negatively with dissolved and total litter P. The relationship was strongest when lignin decomposition rates were compared to litter-microbe C:P imbalances, i.e. the greater the imbalance between resource and consumer C:P became (greater P limitation) the lower lignin decomposition rates became.

Cultivation studies showed that lignin decomposition by fungi is triggered by nitrogen starvation, and that lignin does not provide sufficient energy to maintain the decomposer's metabolism without the use of other organic C i.e. energy sources \cite{Janshekar1988}. Moreover, lignin decomposition was found in wild-type \emph{A. thaliana} litter containing abundant cellulose as a C source, but not in a low-cellulose mutant during a 12-month incubation experiment in a boreal forest \cite{Talbot2011}. In the N- and P-(co-)limited situation commonly encountered during early litter decomposition, we may speculate that lignin is degraded to access additional nutrients (mainly N) or to use a C surplus by decomposing a less C efficient but nutrient enriched substrate (nutrient mining hypothesis). However, a stimulation of lignin decomposition by low P availability or microbial P limitation, as indicated by the strong negative correlations to P pools that we found, has not been reported yet. Though lignified materials have been reported to be N-rich and decomposition of these materials may therefore enhance N supply to microbial communities, lignins are not expected to contain quantitative important amounts of P.

In order to decompose litter lignin and carbohydrates, microbial decomposers rely on the production and excretion of hydrolytic and oxidative extracellular enzymes. While the absolute amounts, in which these enzymes are produced, were largely controlled by N availability, the ratio in which they were produced was strongly related to differences in the ratio of cellulose/lignin decomposition. \cite{Talbot2011} suggested that lignin decomposition comprises a strategy of slow-growing microbes to evade competition through colonizing more lignin-rich and nutrient-poor substrates. Indeed we found lignin decomposition in low quality litter (low N and P) with microbial communities that were subject to large imbalances in C:N and C:P between resource and consumer, pointing to N and P limitation or high N and P uns efficiency of these communities. Low P availability may limit fast growth of microbial populations and select for slow-growing lignin-degrading microbes during early decomposition and provide K-strategists (slow growing on recalcitrant carbon) an advantage over r strategists (fast growing on labile carbon). Indeed we found that lignin decomposition was highest in litter, where resource C:P and N:P were highest, i.e. low P supply may have limited microbial growth generally or the establishment of r strategists in particular.

While the mode of negative P regulation on lignin decomposition remains unknown, we found differences in the composition of the microbial decomposer communities on litter with fast and slow lignin decomposition. Unlike predicted by ecological stoichiometry theory, not bacteria but fungi were more successful in colonizing high N and high P litter during initial decomposition. Fungi colonized litter faster than bacteria and therefore dominated early litter decomposition, however the fungi: bacteria ratios decreased over the entire incubation period pointing to increasing population sizes of bacteria with time. Fungi-rich communities more efficiently used high litter N to produce extracellular enzymes that degrade carbohydrates immediately after inoculation (fungi: bacteria ratios were correlated to litter N 14 days after inoculation) and high litter P to build up microbial biomass on a longer time scale (fungi: bacteria ratios were correlated to litter P after 6 months). Interestingly, bacteria-rich communities (AK) were more active in decomposing lignin than those being dominated by fungi. This does not necessarily indicate that bacteria play the key role in lignin decomposition, though bacteria were also reported to produce oxidative enzymes that can decompose lignified materials in litter \cite{Bugg2011}. However, decreases in fungi/bacteria ratios may be superimposed on the increase of smaller subpopulations of e.g. fungi that are key mediators of lignin decomposition, or alternatively general increases in the size of microbial communities with declining fungi/bacteria ratios may as well mask stable fungal populations when  bacterial abundance increases. The fungal communities were dominated by Ascomycetes (Dothideomycetes, Eurotiomycetes, Leotiomycetes and Sordariomycetes), with smaller contributions by Saccharomycetes and Basidiomycetes (Agaricomycetes and Tremellomycetes). It is particularly the latter, Basidiomycetes, that catalyze the cellulolytic and lignolytic decomposition of dead plant material, however they comprised less that 5\% of the fungal protein ensemble. The bacterial community in contrast was dominated by Proteobacteria (mainly $\gamma$, declining, and $\alpha$- and $\beta$-Proteobacteria, increasing with litter decomposition), Actinobacteria and Bacteroidetes both of which strongly increased with time. Actinobacteria are also known as important decomposers of plant detritus, with the potential to excrete oxidative enzymes and being oligotrophic, and Bacteroidetes also excrete a broad range of hydrolytic enzymes targeting cellulose and other polymers. Since the metaproteomic approach did not find oxidative extracellular enzymes we so far cannot dissect the contributions of bacteria and fungi to the lignin decomposition process. 

While the microbial communities were strictly homeostatic during the first 6 months, substrate stoichiometry had a minor, but significant influence on microbial stoichiometry after 15 months. Together, these changes indicate that the microbial communities were able to compensate for differences in substrate quality by adjusting their C-, N- and P-use efficiency (Mooshammer et al. 2011) which was coupled to differences in substrate preference (lignin/carbohydrate) and occurred at the expense of microbial community growth and overall decomposition speed. However, stoichiometric compensation of the microbial communities was limited after 6-15 months which points to larger stoichiometric differences between the microbial populations dominating the later stage decomposition processes.

%The change in decomposition dynamics further corresponded to changes in soluble organic C and organic C increased by \textgreater 10-fold between 6 and 15 months. While during the first 3 months, soluble organic C concentrations were not (or to a lower extent) correlated to litter N, soluble organic C was strictly correlated with litter N and with actual respiration rates after 6 and 15 months\footnote{stats}. \cite{Klotzbucher2011} suggested a change in decomposition dynamics after 100 to 200 days of incubation, after which lignin decomposition rates decreased due to a lack of labile carbon. They also reported a correlation between respiration rates and extractable C after this change. The authors interpreted this correlation as carbon limitation to respiration, and suggested that lignin decomposition became inhibited under such a limitation by lack of labile C. We found a similar correlation between extractable organic C and respiration after 6 months, but no the inhibition of lignin decomposition since soluble organic C increased during this phase. Moreover, both respiration and organic C concentration were correlated with litter N at this stage. We therefore suggest that litter N concentration was the key control over both dissolved organic C production and respiration, since the process of degrading macromolecular compounds into soluble molecules is conducted by extracellular enzymes and is therefore N sensitive.

\section*{Conclusions}

Our results contradict the traditional concept that lignin decomposition is slow during early litter decomposition. While traditional litter decomposition models propose that lignin decomposition mainly occurs during late decomposition stages, we found that variable but in some cases substantial amounts of lignin were decomposed during the first 6 months. The extent to which lignin was decomposed was controlled by litter P during the first 6 months, but by litter N thereafter as was carbohydrate decomposition. Our results further question that recalcitrance is intrinsic to lignin as a chemical compound, but suggests that lignin decomposition also depends on litter chemistry and environmental conditions, which both affect microbial community structure including the abundance of fungal and bacterial groups that are key to decomposition of plant debris by excretion of hydrolytic and oxidative extracellular enzymes.

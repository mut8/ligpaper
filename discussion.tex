\section*{Discussion}

The experimental approach chosen allowed us to single out the effects of litter quality on the microbial decomposer community and decomposition processes, while excluding effects of fauna, climate and different initial microbial communities. By exploiting intra-specific differences in beech litter stoichiometry, we were able to minimize differences in the chemical composition of initial litter (e.g. similar lignin and cellulose content, table \ref{initstoech}), while exploring the effect of litter nutrient contents on lignin and carbohydrate decomposition. Therefore, we can attribute different rates of carbohydrate and lignin decomposition to the intrinsic qualities of litter collected at different sites, i.e. elemental and stoichiometric composition. Analytical pyrolysis allowed specific determination of litter lignin contents (simultaneously with carbohydrates), avoiding the limitations of common methods for lignin quantification as AUR \cite{Hatfield2005}.

Contradicting traditional concepts of litter decomposition, our results demonstrate that variable amounts of lignin were degraded during the first 6 months of incubation. During this early stage, lignin decomposition rates depended on litter quality (P) and ranged from non-significant to degradation rates similar to bulk carbon mineralization rates (i.e. no discrimination against lignin). We can therefore confirm that early lignin decomposition rates are by far underestimated, as recently proposed by \cite{Klotzbucher2011}, based on a complementary analytic approach. Unlike \cite{Klotzbucher2011}, we found no decrease but constant or increasing lignin decomposition rates during litter decomposition over 15 months. 

Our results provide strong evidence that litter C:N:P stoichiometry and litter element concentrations exerted a major control on the extent of lignin decomposition during the initial decomposition phase. Carbohydrate decomposition was positively correlated with litter N contents and negatively to litter C:N ratios, as were the majority of decomposition processes (mass loss, respiration, potential extracellular enzymatic activities). In contrast, lignin decomposition rates were positively correlated with litter C:P ratios and negatively with dissolved and total litter P. The relationship was strongest when lignin decomposition rates were compared to litter-microbe C:P imbalances, i.e. the greater the imbalance between resource and consumer C:P became (greater P limitation) the lower lignin decomposition rates became. Additionally, we found a marked change in the controls of lignin decomposition during this period. While carbohydrate and lignin decomposition were differently controlled by litter chemistry (N versus P) during the first 6 months, these litter components were decomposed at similar rates thereafter and decomposition rates were only related to litter N availability. 

Cultivation studies showed that lignin decomposition by fungi is triggered by nitrogen starvation, and that lignin does not provide sufficient energy to maintain the decomposer's metabolism without the use of other organic C i.e. energy sources \cite{Janshekar1988}. Moreover, lignin decomposition was found in wild-type \emph{A. thaliana} litter containing abundant cellulose as a C source, but not in a low-cellulose mutant during a 12-month incubation experiment in a boreal forest \cite{Talbot2011}. In the N- and P-(co-)limited situation commonly encountered during early litter decomposition, we may speculate that lignin is degraded to access additional nutrients (mainly N) or to use a C surplus by decomposing a less C efficient but nutrient enriched substrate (nutrient mining hypothesis). However, a stimulation of lignin decomposition by low P availability or microbial P limitation, as indicated by the strong negative correlations to P pools that we found, has not been reported yet. Though lignified materials have been reported to be N-rich and decomposition of these materials may therefore enhance N supply to microbial communities, lignins are not expected to contain quantitative important amounts of P.

In order to decompose litter lignin and carbohydrates, microbial decomposers rely on the production and excretion of hydrolytic and oxidative extracellular enzymes. While the absolute amounts, in which these enzymes are produced, were largely controlled by N availability, the ratio in which they were produced was strongly related to differences in the ratio of cellulose:lignin decomposition. \cite{Talbot2011} suggested that lignin decomposition comprises a strategy of slow-growing microbes to evade competition through colonizing more lignin-rich and nutrient-poor substrates. Indeed we found lignin decomposition in low quality litter (low N and P) with microbial communities that were subject to large imbalances in C:N and C:P between resource and consumer, pointing to N or P limitation. Low P availability may limit fast growth of microbial populations and select for slow-growing lignin-degrading microbes during early decomposition and provide K-strategists (slow growing on recalcitrant carbon) an advantage over r strategists (fast growing on labile carbon). Indeed we found that lignin decomposition was highest in AK litter, where resource C:P and N:P were highest, i.e. low P supply may have limited microbial growth generally or the establishment of r strategists in particular.

Differences in initial lignin contents were marginal (29-31 \% relative peak area), and lignin degradation rates of sites with high lignin contents were not higher than that of sites with low lignin contents. Therefore, in early lignin decomposition was not triggered by critical lignin contents (i.e. lignin limiting the use of other C sources) as is suggested by traditional litter decomposition models. Neither did low lignin decomposition rates result from a lack of metal cofactors of oxidative lignin decay (i.e. Mn or Fe), which were suggested to be rate limiting for late lignin decomposition\cite{Berg2008}. While Mn and Fe concentrations indeed strongly varied between litter collected at different sites, concentrations were lowest in the litter with highest lignin decomposition rates (AK, see Table \ref{initstoech}). Moreover, soluble organic C (``DOC'') was suggested to limiting lignin decomposition since the process of lignin decomposition does not generate sufficient energy to power the metabolism of its decomposers\cite{Klotzbucher2011}. However, soluble organic C  did not control lignin decomposition in this experiment since we found highest (initial) concentrations in the two litter types that showed the highest and the lowest lignin decomposition rates (SW and AK). In contrast, while the initial content of soluble C is not related to litter stoichiometry or respiration, we found a strong correlation of soluble C, litter N after 6 and 15 months. This rather supports a perspective in which soluble C is rapidly used by litter decomposers, while its production is limited by the production of extracellular enzymes and therefore by N availability, which in our experiment also correlated with overall C mineralization rates.

While the mode of negative P regulation on lignin decomposition remains unknown, we found corresponding differences in the composition of the microbial decomposer communities on litter with fast and slow lignin decomposition. Unlike predicted by ecological stoichiometry theory, not bacteria but fungi were more successful in colonizing high N and P litter during initial decomposition. Fungi colonized litter faster than bacteria and therefore dominated early litter decomposition, however the F:B ratios decreased over the entire incubation period pointing to increasing population sizes of bacteria with time. Interestingly, low F:B communities (AK) were more active in decomposing lignin than those being dominated by fungi. This does not necessarily indicate that bacteria play the key role in lignin decomposition, though bacteria were also reported to produce oxidative enzymes that can decompose lignified materials in litter \cite{Bugg2011}. However, decreases in F:B ratios may be superimposed on the increase of smaller subpopulations of e.g. fungi that are key mediators of lignin decomposition, or alternatively general increases in the size of microbial communities with declining fungi/bacteria ratios may as well mask stable fungal populations when  bacterial abundance increases. Furthermore, we are well aware of the short-coming of metaproteomic approaches that under-represent important soil phyla such as uncultivated and unsequenced groups (e.g. Lecanoromycetes). 

The onset of lignin degradation in litter from all sites between 15 months was accompanied by increased abundance of fungal and bacterial taxa associated with late litter decomposition: Complete lignin degradation is most commonly accounted to Basidomycetes \cite{Berg2008}, which however accounted for less than 5 \% of fungal protein in all analyzed samples. Among bacteria, lignin degradation was reported for Actinomycetes, $\alpha$-, and $\gamma$-Proteobacteria \cite{Bugg2011}. The change in controls over lignin degradation  after 6 months (i.e. lignin degradation dependent on litter stoichiometry) was accompanied by increased abundances of Actinomycetes and $\alpha$-Proteobacteria. In contrast, several Ascomycetes classes (Sordariomycetes, Saccheromycetes), which were highly abundant in early decomposition stages and are associated with the rapid degradation of labile C substrates [lit!!!], decreased in their relative abundance. $\gamma$-Proteobacteria abundance was correlated to lignin degradation activity after 6 months of incubation. However, since the metaproteomic approach did not find relevant amounts of oxidative extracellular enzymes we so far cannot dissect the contributions of bacteria and fungi to the lignin decomposition process. 

\section*{Conclusions}
Our results contradict the traditional concept that lignin decomposition is slow during early litter decomposition. While traditional litter decomposition models propose that lignin decomposition mainly occurs during late decomposition stages, we found that variable but in some cases substantial amounts of lignin were decomposed during the first 6 months. We can therefore conclude, that low nutrient levels (mainly P) led to an earlier onset of lignin degradation (Q1). On the other hand, the results of our experiment contradicted our second hypothesis (Q2): We found that regardless of their wider C:N and C:P ratios, fungal communities were more dominant in litter with high litter N and P contents, especially within the first 6 months of decomposition. Regardless of low F:B ratios, P-poor, but bacteria-rich communities were most actively degrading lignin during early decomposition.  In contrast, late lignin decomposition rates were highest in N-rich litter with high F:B ratios, indicating a fundamental change in the controls of lignin decomposition between early and later decomposition stages.













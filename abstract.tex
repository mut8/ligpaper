\section*{Abstract}

Lignin is a major component of plant litter and is considered highly resistant to decomposition. Polymeric carbohydrates, in contrast, are more easily accessible carbon sources. We studied the decomposition rates of these two compound classes, to which extent they are controlled by litter C:N:P stoichiometry, and whether this control changes over time. Therefore, we conducted a 15-months mesocosm experiment under controlled climatic conditions, comparing beech litter of different N and P contents, which was sterilized and re-inoculated with a litter/topsoil mixture from one of the sites to ensure identical microbial communities at the start of the experiment. Lignin and carbohydrate decomposition rates were estimated for 2 periods (0-6 months and 6-15 months) by pyrolysis-GC/MS.

Positive correlations of carbohydrate decomposition rates with litter N content were found during the entire experiment. Lignin decomposition rates during the initial period were highly variable and negatively correlated to litter P content and positively correlated to the microbial P demand (C:P\textsubscript{litter}/C:P\textsubscript{microbial}). During the later stage, lignin decomposition rates were positively correlated to N contents, respiration, and carbohydrate decomposition. Initial lignin decomposition rates were highest in litter with low fungi/bacteria ratios, which occurred in N and P poor litter.

Our results showed that a substantial amount of lignin can be degraded during early decomposition. In the present study, early lignin decomposition was coupled to low N and P availability, and the establishment of K-strategist microorganisms. However, early lignin decomposition rates did not depend on fungi, which are commonly assumed to mediate lignin decomposition, or stoichiometric conditions that favor fungal growth. 
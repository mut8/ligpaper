\section*{Abstract}

Lignin is a major component of plant litter and is considered highly resistant to decomposition. Carbohydrates, in contrast, are more easily degraded. We studied the decomposition rates of these two compound classes, to which extent they are controlled by litter C:N:P stoichiometry. We report results from a 15-months mesocosm experiment with beech litter of different N and P contentsunder controlled climatic conditions. Litter was sterilized and re-inoculated prior to the experiment to investigate the effect of litter stoichiometry on an identical initial decomposer community. Lignin and carbohydrate decomposition rates were determined by pyrolysis-GC/MS for 2 periods (0-6 months and 6-15 months), the composition of the microbial community vas monitored via the litter metaproteome.

Rates of lignin decomposition were highly variable during the first six months, and lignin decomposition ranged from insignificant composition to composition at bulk C mineralization rates. Between 6 and 15 months, lignin was degraded at bulk mass loss rates regardless of the litter composition, however, different lignin contents aquired within the first 15 months remained in place.

Fungi were dominant on all litter, but fungi:bacteria ratios were highest on high-nutrient litter, leading to a negative correlation between litter and microbial stoichiometry and high differences between litter and microbial C:N and C:P ratios on nutrient poor litter. These differences (C:P\textsubscript{litter}/C:P\textsubscript{microbial}) were highly correlated to early lignin decomposition rates. Furthermore, lignin degrading communities had lower bacteria:fungi ratios, and were enriched in $\gamma$-proteobacteria. 

Our results indicate that - contradicting common models - significant amounts of lignin were be degraded during early decomposition in low nutrient litter. In sum, we can demonstrate that litter quality profoundly affects the lignin decomposition via the composition of the decomposer community, shaping the stoichiometric constrains of the decomposition process, and controlling lignin decomposition during early decomposition. 
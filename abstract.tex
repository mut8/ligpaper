\section*{Abstract}

Lignin is a major component of plant litter and is considered highly resistant to decomposition. Carbohydrates, in contrast, are more easily degraded. We studied the decomposition rates of these two compound classes, and to which extent they are controlled by litter C:N:P stoichiometry. 

Herein we report results from a 15-months mesocosm experiment under controlled climatic conditions with beech litter of different N and P contents. Litter was sterilized and re-inoculated prior to the experiment to minimize differences in the initial microbial community, but study the effect of N and P contents on identical initial decomposer communities. Lignin and carbohydrate decomposition rates were determined by pyrolysis-GC/MS for 2 periods (0-6 months and 6-15 months), the composition of the microbial community was monitored via metaproteome analysis.

Different in litter nutrient contents led to the establishment of different decomposer communities. Fungi were dominant on all litter, but fungi:bacteria ratios were highest on high-nutrient litter, leading to a negative correlation between litter and microbial stoichiometry and high differences between litter and microbial C:N and C:P ratios on nutrient poor litter.

Rates of lignin decomposition were highly variable during the first six months, ranging from insignificant amounts decomposed to decomposition at bulk C mineralization rates. Between 6 and 15 months, lignin was degraded at bulk mass loss rates independent of the litter nutrient contents, however, different lignin contents aquired within the initial 6 months remained in place. Early lignin degradation rates were highest in litter with low fungi:bacteria ratios, and were correlated to differences between litter and microbial stoichiometry (C:P$_{litter : microbial}$ and C:P$_{litter : microbial}$). Lignin degrading communities were enriched in $\gamma$-proteobacteria after 6 months incubation.

Our results indicate that - contradicting common models - significant amounts of lignin were be degraded during early decomposition in low nutrient litter. We demonstrate that litter quality profoundly affects the lignin decomposition via the composition of the decomposer community. Even though bacterial biomass is enriched in N and P, communities low nutrient litter were enriched in bacteria. This led to higher differences between litter and microbial stoichiomety, a possible control over lignin degradation during early decomposition. 
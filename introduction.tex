\section*{Introduction}

Plant litter is quantitatively dominated by macromolecular compounds. In foliar litter, lignin and carbohydrate polymers together make up 40-60\% of litter dry mass\cite{Berg2008}, while leachable substances ("DOM") account for only 1.5-6\% \cite{Don2005}. The breakdown of these high molecular weight compounds into smaller molecules accessible to microbes is mediated by extracellular enzymes and is therefore considered to be the rate limiting for decomposition processes \cite{Sinsabaugh2010}

Common models of litter decomposition \cite{Berg1980, Couteaux1995, Moorhead2006, Adair2008} assume that organic compounds in litter form up to three independent pools of increasing recalcitrance, i.e. (1) soluble compounds, (2) cellulose and hemi-celluloses, and (3) lignin and waxes (cutin and suberin). Soluble compounds are most accessible to microbes and are usually consumed first, followed by regular polymers, such as cellulose. Lignin is not degraded until accumulated to a certain, critical level when it inhibits the degradation of less recalcitrant compounds. Most studies quantified these pools by gravimetric determination of the amount of cellulose, hemi-celluloses and lignins after sequential extractions with selective solvents. These methods were repeatedly criticized for being unspecific for lignin determination\cite{Hatfield2005}. When analyzed with alternative methods (NMR, CuO-oxidation, Pyrolysis-GC/MS), extracted lignin fractions were shown to contain also many other substances\cite{Preston1997}.

Recent studies based on more specific methods to determine litter lignin contents question the intrinsic recalcitrance of lignin. Isotope labeling experiments with soils and litter/soil mixtures, undertaken both in-situ and under controlled conditions,  revealed mean residence times of lignin in soils in the range of 10-50 years, much less then expected and shorter than that of bulk soil organic matter\cite{Amelung2008, Thevenot2010a, Bol2009}. While the ability to completely degrade lignin was traditionally  attributed exclusively to Basidomycetes, it has been demonstrated for several bacterial taxa over the last years\cite{Bugg2011}.

For leaf litter, lignin decomposition even at early stages of litter decay and lignin decomposition rates that decreased during decomposition were recently reported by Klotzb\"{u}cher and colleagues \cite{Klotzbucher2011}. Based on these results, they proposed a new concept for lignin degradation in which fastest lignin degradation occurs during early litter decomposition when the availability of labile carbon sources is high. Lignin decomposition during late decomposition, in contrast, is limited by the availability of easily assimilated C and therefore slows down. Additionally, the decomposition of lignin may also be dependent on the nutrient content of the litter and thus the nutritional status of the microbial community. During radical polymerization, significant amounts of cellulose and protein are incorporated into lignin structures \cite{Achyuthan2010}. In isolated lignin fractions from fresh beech litter, N contents twice as high as in bulk litter were found \cite{Dyckmans2002}. It was therefore argued that, while yielding little C and energy, lignin decomposition makes protein accessible to decomposers that is occluded in plant cell walls, and that lignin decomposition is therefore not driven by C but by the N demand of the microbial community ("Nitrogen mining theory", \cite{Craine2007}). 

In favor of the N mining theory, fertilization experiments indicated N exerts an important control on lignin degradation: N addition increased mass loss rates in low-lignin litter while slowing down decomposition in lignin-rich litter \cite{Knorr2005} and decreased the activity of lignolytic enzymes in forest soils \cite{Sinsabaugh2010}. Moreover, cellulose triggered a stronger priming effect in fertilized than in unfertilized soils indicating that the mineralization of recalcitrant carbon may be controlled by an interaction of N and labile C availability \cite{Fontaine2011}.

Addition of N has a different effect on litter decomposition than varying N levels in the litter \cite{Talbot2011}. This is due to the fact that leaf litter N is stored in protein and lignin structures and not directly available to microorganisms, while fertilizer N is added in the form of readily available inorganic N (ammonium, nitrate or urea). N-fertilization experiments can thus simulate increased N-deposition rates but not the effect of litter N on decomposition processes.

Our study therefore aimed at analyzing the effect of variations in beech litter nutrient (N and P) content and stoichiometry (C:N and N:P ratios) on lignin and carbohydrate decomposition rates. Towards this end, we followed the breakdown of lignin and polymeric carbohydrates by pyrolysis-GC/MS (pyr-GC/MS) during a mesocosm experiment under constant environmental conditions over a period of 15 month. In order to exclude effects resulting from different initial microbial communities, we sterilized beech litter samples from 4 different locations in Austria and re-inoculated them prior to the experiment with an litter/top-soil inoculum from one of the sites.  
Additionally, we analyzed the micorbial community in a subset of the samples using a metaproteomic approach.

With the experiment, we addressed the following questions:

(Q1) Is lignin decomposition delayed until late decomposition stages or are significant amounts of lignin already degraded during early litter decomposition, and is the timing of lignin decomposition dependent on litter stoichiometry? We hypothesized, that ligin decomposition is initially slower in litter with a narrow C:N ratio (higher availability of assimilable nitrogen), than in litter with a high C:N ratio.

(Q2) Are high lignin degradation rates related to a higher fungal activity? We hypothesized that wider C:N and C:P ratios favor lignin degradation by fungi while narrow C:N and C:P ratios favor carbohydrate degradation by bacteria. 
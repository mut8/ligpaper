\section*{Discussion}

Different in litter qualities led to different mass loss rates and the development of of different decomposer communities from the same inoculate. Lignin decomposition was highly variable between litter of different quality within the first 6 month; it's decomposition ranged from non-detectable (SW litter) to decomposition at bulk carbon loss rates (ie. no lignin accumulation, AK litter). This provides further evidence that the use of extractive methods to measure lignin contents led to an underestimation of early lignin degradation rates, as recently suggested \cite{Klotzbucher2011}, and that substancial amounts of lignin can be degraded during early decomposition. In contrast, between 6 and 15 months, lignin was degraded at the same rate as bulk carbon in all litter, regardless of litter quality. During this time, the different lignin contents aquired within the first six months remained in place, but lignin contents no further increased. 

We chose our collection sites to provide litter with different N and P contents since we focused on the effect of these nutrients on decomposition. We found positive correlations between litter N and bulk decomposition parametres like carbon mineralization rates and extracellular enzymatic activities, indicating litter decomposition was limited by litter N. No such correlation was found for litter P contents. However, AK litter, which had low contents in both N and P, had a lower C mineralization rate than KL and OS, which were had lower contents of N or P, respectively, than AK, suggesting a co-limitation of both elements. The use of litter of a single species from different sites minimized differences in other litter traits, and eg. initial carbohydrate and lignin contents of all samples fell in a narrow range for litter from all sites. However, litter N and P contents are also proxies for other litter traits not directly measurable (e.g. leaf morphology), which resulted from the plant's response to 
nutrient availability (e.g. for low P adaptation see \cite{Vance2003}). N and P were also demostrated to be correlated to a wide area of leaf traits in plants (ie. \cite{Wright2005}), and such leaf traits were successfully used to predict litter decomposeability in the past\cite{Cornelissen1997}. 

The composition of the microbial community changed with both by time (i.e. succession) and collection site (i.e. litter quality). While all samples measured were dominated by fungi, fungi:bacteria ratios decreased over time and were higher in nutrient-rich litter than in nutrient-poor litter (SW \textgreater \textgreater  AK). Our results contradicted the often-cited predictions that higher N and P contents would favor bacterial over fungal growth because bacterial biomass has lower C:N and C:P ratios than fungal biomass \cite{Hodge2000}. In contrast, we found fungi : bacteria ratios were higher in nutrient-rich litter. Similar observations were reported by G\"{u}sewell and Gessner 2009, who suggested that bacteria compensate N deficiency by heterotrophic N fixation, and therefore colonize low-N litter more successfully. However, microbial decomposers excrete important amounts of N as extracellular enzymes, which further raise their N demand; a factor not represented in the biomass C:N ratios. The higher 
abundance of bacteria on low nutrient litter would also be explained if fungi produced more extracellular enzymes per biomass than bacteria, therefore have creating a more narrow C:N demand than bacteria even though their biomass has a wider C:N ratio. In result, bacteria-rich decomposers with more narrow C:N and C:P biomass developed on low-nutrient sites, with more narrow C:N and C:P ratios, further increasing difference between microbial and litter stoichiometry. To consider both litter and microbial stoichiometry, we used C:X$_{litter : microbial}$ ratios as integrated measure for nutrient availability to nutrient demand.

Litter quality controlled the composition of the microbial communities, and microbial biomass stoichiometry, which influenced the stoichiometric offset between resource (litter) and consumer (microbial biomass) that decomposers had to overcome. Both community composition, and C:X$_{litter : microbial}$ and were correlated to the rate of early lignin decomposition; Litter with high fungi:bacteria ratios and lower differences between litter and microbial stoichiometry accumulated more lignin. Our results therefore indicate that lignin degradation is associated with bacteria-rich degrader communities, a low availability of nutrients to decomposers, or both. In contrast, we can exclude that lignin decomposition was triggered by critical lignin contents or inhibited by insufficient Mn (highest lignolytic activity in litter with lowest contents of lignin and Mn), as suggested for late lignin decomposition \cite{Berg2008}. 

Traditionally, the capability to completely degrade lignin was exclusively attributed to Basidomycota fungi \cite{Berg2008}. However, fungi:bacteria ratios were lower in lignin degrading litter and Basidomycota produced less than 5\% of fungal protein. Over the last years, lignin degradation was also demonstrated for several bacterial taxa (eg. actinomycetes, $\alpha$-, and $\gamma$-proteobacteria \cite{Bugg2011}). Of these three taxa, we found one ($\gamma$-proteobacteria) correlated to lignolytic activities after 3 and 6 months. The other two taxa (actinomycetes and $\alpha$-proteobacteria) were enriched after 15 months in all litter types, when lignin decomposition was found in litter from all sites (ie.  independent from litter quality). However, the metaproteomic analysis only sporadically detected lignolytic enyzmes, we can not attribute lignolytic activities to specific taxa at this time. Nevertheless, our data indicates that corresponding trends for lignin degradation and fungal : bacterial 
protein abundance both along succession and between litter from different sites after 6 months (higher lignolytic activity at low fungi:bacteria ratios). 

Lignin degradation in the first 6 months was best predicted by C:P$_{litter : microbial}$ and C:N$_{litter : microbial}$, ie. more lignin was degraded when the difference between litter and microbial stoichiometry was higher. C:P$_{litter : microbial}$ and C:N$_{litter : microbial}$ were intercorrelated, so we can not differentiate the effects of N and P. However, lignin decomposition rates were negatively correlated to litter P, but not N contents and positively correlated to microbial biomass P contents, while carbohydrate decomposition was positively correlated to litter N. This would indicating a differential control of lignin (stimulated by P demand) and carbohydrate (stimulated by N availability) decomposition. Litter which rapidly degraded lignin (AK and KL) net-mobilized insoluble P into rapidly cycled P forms (soluble and microbial), while in slowly lignin degrading litter (SW and OS) microbial P originated only from soluble P (fig. \ref{fig_phos}B). This indicates that lignin degradation increased the mobilization of P from insoluble litter biomass , and explains higher lignin degradation rates in litter with high microbial P demand. Such a nutrient mobilization by lignin degradation is assumed for N \cite{Craine2007}: Lignin (like humic compounds) occludes important amounts of protein during polymerization \cite{Achyuthan2010, Dyckmans2002}, which is available only after lignin degradation. Therefore, the degradation of complex carbon sources was proposed to constitute a strategy of N sequestration (N mining theory, \cite{Moorhead2006, Craine2007}). However, ambivalent results are reported for whether P is protected in lignin and humic compounds, and whether P demand triggers the decomposition of complex carbon compounds \cite{Craine2007, Osono2004}.

The degradation of lignin and carbohydrate polymers depends on the excretion of different extracellular enzymes. Their production is N intensive, therefore a trade-off exists between the production of cellulolytic and lignolytic enzymes \cite{Sinsabaugh2010}. Lignin decomposition was also suggested to allow decomposers direct competition by the early colonization of lignin rich sites \cite{Treseder 2011}. We found higher activities of both cellulolytic and lignolytic enzymes in N-rich litter, but their ratio was well correlated to lignin : carbohydrate decomposition. Lignin degradation yields less C and energy than carbohydrates degradation, but might provides additional N. When C:X$_{substrate : consumer}$ is low, as it was the case for lignin degrading litter, additional carbon can not be used by decomposers to build up biomass. Therefore, the observed increase in lignolytic activities might result from a microbial strategy to optimize N allocation between cellulolytic and lignolytic enzyme 
systems when additional C can not be utilized by the decomposer due to a lack of nutrients. 

In summary, litter quality exercised a profound control on the litter decomposition process, including community composition and lignin accumulation. Lignin decomposition within the first six months was highly variable between litter of of the same species but different in nutrient contents, and ranged from no lignin degradation to lignin degradation at bulk litter C loss rates. Lignin decomposition was not coupled to fungi-rich decomposer communities, indicating important bacterial contributions to lignin decomposition. Early lignin decomposing litter was characterized by nutrient-rich microbial biomass on low-nutrient litter, ie. a high difference between litter and decomposer stoichiometry, low fungi:bacteria ratios and an elevated abundance of $\gamma$-proteobacteria peptides. Different lignin contents acquired during early decomposition remained in place, potentially affecting late decomposition and humification.	
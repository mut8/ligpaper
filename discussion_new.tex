\section*{Discussion}

We used beech litter collected at different sites to study the effect of litter quality on decomposition. Differences in litter quality led to different mass loss rates and the development of of different decomposer communities from the same inoculate. Lignin decomposition  was highly variable between litter of different quality within the first 6 month; lignin decomposition rates ranged from no detectable degradation (SW litter) to decomposition at bulk carbon loss rates (ie. no lignin accumulation, AK litter). Our results therefore provide further evidence that the use of extractive methods two measure lignin contents led to an underestimation of early lignin degradation rates, as recently suggested \cite{Klotzbucher2011}, and that substancial amounts of lignin can be degraded during early decomposition. In contrast, between 6 and 15 months, lignin was degraded at the same rate as bulk carbon in all litter, regardless of litter quality. During this time, the different lignin contents aquired within the first six months remained in place, but lignin contents no further increased.  

Our study was designed to investigate the effect of litter N and P contents on decomposition, and we chose our collection sites to provide litter with different N and P contents. We found positive correlations between litter N and bulk decomposition parametres like carbon mineralization rates and extracellular enzymatic activities, indicating litter decomposition was limited by litter N. No such correlation was found for litter P contents. However, AK litter, which had low contents in both N and P, had a lower C mineralization rate than KL and OS, which were had lower contents of N or P, respectively, than AK, suggesting a co-limitation of both elements. The use of litter of a single species from different sites minimized differences in other litter traits, and eg. initial carbohydrate and lignin contents of all samples fell in a narrow range for litter from all sites. However, litter N and P contents are also proxies for other litter traits not directly measurable (e.g. leaf morphology), which resulted from the plant’s response to nutrient availability (e.g. for low P adaptation see \cite{Vance2003}). N and P were also demostrated to be correlated to a wide area of leaf traits in plants (ref\footnote{good ref for leaf economy spectrum needed!}), and such leaf traits were successfully used to predict litter decomposeability in the past\cite{Cornelissen1997}. 

The composition of the microbial community changed with both by time (i.e. succession) and collection site (i.e. litter quality). While all samples measured were dominated by fungi, fungi:bacteria ratios decreased over time and were higher in nutrient-rich litter than in nutrient-poor litter (SW >> AK). Our results contradicted the often-cited predictions that higher N and P contents would favor bacterial over fungal growth because bacterial biomass has lower C:N and C:P ratios than fungal biomass \cite{Hodge2000}. In contrast, we found fungi : bacteria ratios were higher in nutrient-rich litter. Similar observations were reported by Güsewell and Gessner 2009, who suggested that bacteria compensate N deficiency by heterotrophic N fixation, and therefore colonize low-N litter more successfully. However, microbial decomposers excrete important amounts of N as extracellular enzymes, what further raise their N demand; a factor not represented in the biomass C:N ratios. The higher abundance of bacteria on low nutrient litter would also be explained if fungi produced more extracellular enzymes per biomass than bacteria, therefore have creating a more narrow C:N demand than bacteria even though their biomass has a wider C:N ratio. In result, bacteria-rich decomposers with more narrow C:N and C:P biomass develloped on low-nutrient sites, with more narrow C:N and C:P ratios, further increasing difference between microbial and litter stoichiometry. To consider both litter and microbial stoichiometry, we used C:X$_{litter}$ : C:X$_{consumer}$ ratios as integrated measure for nutrient availability to nutrient demand.



%Correlative analysis indicated a differential control over lignin and carbohydrate decomposition: Carbohydrate loss rates were coupled to bulk mineralization, while lignin decomposition during the later period (6-15 months) controlled by litter N contents (as are bulk litter and carbohydrate loss rates), lignin degradation rates during the first 6 months (and the extend of lignin accumulation over the whole decomposition process) were differentially regulated. 

% 

%We found a different trend during later decomposition (6 - 15 months), when contents of lignin and carbohydrates remained constant in all litter (ie. lignin and carbohydrates decomposed at bulk carbon loss rates) and litter quality affected bulk C mineralization rates, but did not lead to further shifts in the chemical composition of the litter.

%
Litter quality controlled the commposition of the microbial communities, and thereby the difference between litter and microbial stoichiometry, both of which were correlated to the extend of early lignin decomposition. Litter which accumulated more lignin had microbial communities with high fungi:bacteria ratios and lower differences between litter and microbial stoichiometry. Our results therefore indicate that lignin degradation is associated with bacteria-rich communities, a low availability of nutrients to decomposers, or both. 

Traditionally, the capability to completely degrade lignin was exclusively attributed to Basidomycota fungi \cite{Berg2008}. However, Basidomycota made up less than 5\% of fungal protein. Over the last years, the capeability to completely degrade lignin was demonstrated for several bacterial taxa (eg. actinomycetes, $\alpha$-, and $\gamma$-proteobacteria \cite{Bugg2011}). Of these three taxa, we found one ($\gamma$-proteobacteria) correlated to lignolytic activities after 3 and 6 months. The other two taxa (actinomycetes and $\alpha$-proteobacteria) were enriched after 15 months in all litter types, when lignin decomposition was found in litter from all sites (ie.  independent from litter quality). However, since our metaproteomic analysis only sporadically detected lignolytic enyzmes, we can not attribute lignolytic activities to specific taxa at this time. Nevertheless, our data indicates that corresponding trends for lignin degradation and fungal : bacterial protein abundance both along succession and between litter from different sites after 6 months (higher lignolytic activity at low fungi:bacteria ratios). Overall, lignin decomposition was not coupled to fungal abundance, hinting that bacterial contributions to lignin decomposition might be greater than traditionally assumed. Also, the activity of $\gamma$-proteobacteria deserves special attention in further studies on early litter decomposition. 

%Early lignin degradation increased with the difference between litter and microbial stoichiometry (both C:N$_{litter}$ : C:N$_{microbial}$ and C:P$_{litter}$ : C:P$_{microbial}$). These stoichiometric differences were higher for the bacteria-enriched lignin decomposing communities, due to both low-nutrient litter and nutrient-rich microbial biomass. Litter with early lignolytic activity was therefore characterized by different decomposer community compositions and a lower nutrient availability than non-lignolytic litter.and C:N$_{litter}$ : C:N$_{microbial}$

%However, a greater influence of P availability can be expected during early decomposition when nutrients are net-immobilized to cover the nutrient demand of an increasing microbial biomass \cite{Manzoni2010}. Also, more rapidly dividing microorganisms have a more narrow N:P demand than slow growing microorganisms (\cite{Keiblinger2010}), therefore the importance of P is higher during litter colonization.


Lignin degradation in the first 6 months was positively C:P$_{litter}$ : C:P$_{microbial}$ and C:N$_{litter}$ : C:N$_{microbial}$, ie. more lignin was degraded when the difference between litter and microbial stoichiometry was higher. Differentiating between the effects of these two stoichiometric ratios was difficult, since both were intercorrelated. Lignin decomposition rates were also negativly correlated to litter P, but not N contents and positively correlated to microbial biomass P contents, while carbohydrate decomposition was positively correlated to litter N. This would indicating a differential control of lignin (inhibited by P) and carbohydrate (stimulated by N) decomposition. Also, actively lignin degrading litter (AK and KL) net-mobilized insoluble litter P during this time, while in slowly lignin degrading litter (SW and OS) an accumulation of microbial P led to a decrease in soluble P. This suggests that lignin degradation increased the mobilisation of P from insoluble litter biomass into the rapidly recycled P fractions (biomass and disolved), and would explains higher lignin degradation rates in litter with high microbial P demand. However, such a nutrient mobilization by lignin degradation is commonly assumed for N, but not for P \cite{Craine2007}: Lignin (like humic compounds) does not provide sufficient C and energy to sustain the decomposers metabolism, but occludes important amounts of protein during polymerization \cite{Achyuthan2010}, which is available only after lignin degradation. Therefore, lignin degradation was proposed to constitute a strategy of N sequestration (N mining theory, \cite{Sinsabaugh2006, Craine2007}). 

The degradation of lignin and carbohyradate polymers depends on the excretion of different extracellular enzymes. Their production is N intensive, therefore a trade-off exists between the production of cellulolytic and lignolytic enyzmes \cite{Sinsabaugh2010}. Lignin decomposition was also suggested to allow decomposers direct competition by the early colonization of lignin rich sites \cite{Treseder 2011}. We found higher activities of both cellulolytic and lignolytic enzymes in N-rich litter, but their ratio was well correlated to lignin : carbohydrate decomposition. Lignin degradation yields less C and energy than carbohydrates degradation, but might provides additional N. When C:X$_{substrate}$ : C:X$_{consumer}$ is low, as it was the case for lignin degrading litter, additional carbon can not be used by decomposers to build up biomass. Therefore, the observed increase in lignolytic activities might result from a microbial strategy to optimize N allocation between cellulytic and lignolytic enzyme 
systems when additional C can not be untilized by the decomposer due to a lack of nutrients. 

%However, differences in initial lignin contents were marginal, and lignin degradation rate of sites with high lignin contents were lower than those of sites with low lignin contents. Lignin decomposition was therefore not triggered by elevated lignin contents that limit the use of other C sources as suggested by traditional models \cite{}. While Mn and Fe concentrations strongly varied between litter collected at different sites, however, concentrations were lowest in the litter with highest lignin decomposition rates (AK, see Table \ref{initstoech}), so Lignin decomposition rates were not limited by a lack of these these elements. Moreover, soluble organic C (“DOC”) was recently suggested to limit lignin decomposition since the process of lignin decomposition does not generate sufficient energy to power the metabolism of its decomposers (Klotzbücher 2011). However, we found highest (initial) DOC concentrations however were found in the two litter types that showed the highest and the lowest lignin decomposition rates (SW and AK). % a sentence of N & P fertilization vs. N & P variance in litter?


In summary, litter quality had profound influence on the litter decomposition process, including community decomposition and lignin accumulation. Lignin decomposition within the first six months was highly variable between litter of of the same species but differnet in nutrient contents, and ranged from no lignin degradation to lignin degradation at bulk litter C loss rates. It was associated with nutrient-rich microbial biomass on low-nutrient litter, ie. a high difference between litter and decomposer stoichiomtry, low fungi:bacteria ratios and an elevated abundance of $\gamma$-proteobacteria peptides. Differences in lignin contents aquired within the first six months remained in place during later decomposition, although - regardless of litter quality - lignin was not further accumulated between 6 and 15 months.
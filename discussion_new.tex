\section*{Discussion}

We used beech litter collected at different sites to study the effect of litter quality on decomposition. Differences in litter quality led to different mass loss rates, different extends of lignin accumulation during early decomposition, and the development of of different decomposer communities from a common inoculate. Our study was designed to investigate the effect of litter N and P, and we chose our collection sites to provide litter with different N and P contents. Positive correlations between litter N and bulk decomposition parametres like carbon mineralization rates and extracellular enzymatic activities indicate a N limitation. No such correlation was found for litter P contents. However, AK litter, which had low contents in both N and P, had a lower C mineralization rate than KL and OS, which were had lower contents of N or P, respectively, than AK, suggesting a co-limitation of both elements. The use of litter of a single species from different sites minimized differences in other 
litter traits, and eg. initial carbohydrate and lignin contents of all samples fell in a narrow range for litter from all sites. However, litter N and P contents are also proxies for other litter traits not directly measurable (e.g. leaf morphology), which resulted from the plant’s response to nutrient availability (e.g. for low P adaptation see \cite{Vence2003}). N and P were also demostrated to be correlated to a wide area of leaf traits in plants (ref\footnote{good ref for leaf economy spectrum needed!}), and leaf traits were successfully uised to predict litter decomposeability \cite{Cornelissen1997}. % a sentence of N & P fertilization vs. N & P variance in litter?

The composition of the microbial community changed with both by time (i.e. succession) and collection site (i.e. litter quality). While all samples measured were dominated by fungi,  fungi:bacteria ratios decreased over time and were higher in nutrient-rich litter than in nutrient-poor litter (SW >> AK). Our results contradicted the often-cited predictions that higher N and P contents would favor bacterial over fungal growth because bacterial biomass has lower C:N and C:P ratios than fungal biomass \cite{Hodge2000}. In contrast, we found fungi : bacteria ratios were higher in nutrient-rich litter. Similar observations were reported by Güsewell and Gessner 2009, who suggested that bacteria compensate N deficiency by heterotrophic N fixation. Also, quantitative important amounts of N are excreted as extracellular enzymes by litter decomposers and further raise microbial N demand; a factor not represented in the biomass C:N ratios. Fungi might have a higher N demand than bacteria if they produce a higher amount of enzymes per biomass, even though their biomass has a wider C:N ratio. Increased enyzme excretion might also constitute a strategy of decomposers to addopt to substrates with narrow C:N ratios, i.e. instead of mineralizing surplus N decomposers increse the production of extracellular enzymes. Furthermore, adaption to low nutrient concentrations does not necessarily include a change of microbial biomass stoichiometry (which is determined by evolutionary highly conservative traits like cell wall composition), but can also be achieved by physiological adaptions (e.g. lower concentration optima of membran nutrient transporters). In result, bacteria-rich decomposers with narrow C:N and C:P biomass develloped on low-nutrient sites, with more narrow C:N and C:P ratios, further increasing difference between microbial and litter stoichiometry. To consider both litter and microbial stoichiometry, we used C:X$_{litter}$ : C:X$_{consumer}$ ratios as integrated measure for nutrient availability to nutrient demand.

Lignin decomposition within the first 6 month was highly variable between litter of different quality; lignin decomposition rates ranged from no detectable degradation (SW) to decomposition at bulk carbon loss rates (ie. no lignin accumulation, AK). I contrast, between 6 and 15 months, lignin was degraded at the same rate as bulk carbon in all litter, regardless of litter quality, however, different extends of lignin accumulation aquired within the first six months remained in place. Our results therefore provide further evidence that the use of extractive methods two measure lignin contents led to an underestimation of early lignin degradation rates, as recently suggested \cite{Klotzbucher2011}. 

%Correlative analysis indicated a differential control over lignin and carbohydrate decomposition: Carbohydrate loss rates were coupled to bulk mineralization, while lignin decomposition during the later period (6-15 months) controlled by litter N contents (as are bulk litter and carbohydrate loss rates), lignin degradation rates during the first 6 months (and the extend of lignin accumulation over the whole decomposition process) were differentially regulated. 

%Differences in initial lignin contents were marginal, and lignin degradation rate of sites with high lignin contents were lower than those of sites with low lignin contents. Lignin decomposition was therefore not triggered by elevated lignin contents that limit the use of other C sources as suggested by traditional models [ref]. While Mn and Fe concentrations strongly varied between litter collected at different sites, however, concentrations were lowest in the litter with highest lignin decomposition rates (AK, see Table ref), so Lignin decomposition rates were not limited by a lack of these these elements. Moreover, soluble organic C (“DOC”) was recently suggested to limit lignin decomposition since the process of lignin decomposition does not generate sufficient energy to power the metabolism of its decomposers (Klotzbücher 2011). However, we found highest (initial) DOC concentrations however were found in the two litter types that showed the highest and the lowest lignin decomposition rates (SW and AK). We found a different trend during later decomposition (6 - 15 months), when contents of lignin and carbohydrates remained constant in all litter (ie. lignin and carbohydrates decomposed at bulk carbon loss rates) and litter quality affected bulk C mineralization rates, but did not lead to further shifts in the chemical composition of the litter.

%
Our results indicate that litter quality controled the microbial communities on litter, and the offset between litter and microbial stoichiometry, both of which were correlated to the extend of early lignin decomposition. Litter with high fungi : bacteria ratios accumulated more lignin, ie. bacteria rich communities were more actively degrading lignin. Traditionally, the capability to completely degrade lignin was exclusively attributed to Basidomycota fungi \cite{Berg2008}. However, Basidomycota made up less than 5\% of fungal protein. Furthermore, we found faster lignin degradation in litter with bacteria-rich communities. Over the last years, the capeability to completely degrade lignin was demonstrated for several bacterial taxa (including actinomycetes, $\alpha$-, and $\gamma$-proteobacteria \cite{Bugg2011}. Of these three taxa, we found one ($\gamma$-proteobacteria) correlated to lignolytic activities after 3 and 6 months. The other two taxa (actinomycetes and $\alpha$-proteobacteria) were enriched after 15 months in all litter types, when lignin decomposition was found in litter from all sites (ie.  independent from litter quality).

%Early lignin degradation increased with the difference between litter and microbial stoichiometry (both C:N$_{litter}$ : C:N$_{microbial}$ and C:P$_{litter}$ : C:P$_{microbial}$). These stoichiometric differences were higher for the bacteria-enriched lignin decomposing communities, due to both low-nutrient litter and nutrient-rich microbial biomass. Litter with early lignolytic activity was therefore characterized by different decomposer community compositions and a lower nutrient availability than non-lignolytic litter.and C:N$_{litter}$ : C:N$_{microbial}$

Early lignin degradation rates were also correlated to litter and microbial stoichiometry. After 6 months, we found negative correlation between lignin degradation rates and litter P contents (and dissolved PO4), but not with litter N contents. Also, Lignin degradation was positively correlated to microbial biomass P contents. We found the best correlation to the C:P$_{litter}$ : C:P$_{microbial}$, which accounts for both nutrient availability and nutrient demand. In lignin degrading litter (AK and KL) insoluble P was rapidly incorporated into the microbial biomass, while soluble P contents increased (ie. insoluble P was mobilized). In contrast, microbial P contents pools increased more slowly in sites that did degraded little or no lignin (OS and SW), and microbial P was taken up from a shrinking soluble P pool, ie. not from insoluble P. Our results therefore suggest that lignin degradation increases the mobilisation of P from insoluble litter biomass, which would also explain higher lignin degradation rates in litter with high microbial P demand. However, such a nutrient mobilization by lignin degradation is commonly assumed for N, but not for P. Lignin does not provide sufficient C and energy to sustain the decomposers metabolism, but occludes important amounts of protein during polymerization (ref), which is available only after lignin degradation. Therefore, lignin degradation was proposed to constitute a strategy of N sequestration (N mining theory, \cite{Sinsabaugh2006, Craine2007}). In contrast, no such relation is known for P.  However, the effect of P on litter chemistry is less investigated than the effect of N, and nutrient dynamics of early decomposition are substancially different from late decomposition stages. N and P are net-immobilized during early decomposition to cover the rising nutrient demand of an increasing microbial biomass, but available in surplus and therefore net-mineralized during late decomposition \cite{Manzoni2010}. Furthermore, more rapidly dividing microorganisms have a more narrow N:P demand than slow growing microorganisms (\cite{Keiblinger2010}), therefore P is of increased importance during litter colonization. \footnote{come back to leaf traits/scloromorphy under low P here?}

The degradation of lignin and carbohyradate polymers depends on the excretion of different extracellular enzymes. Their production is N intensive, therefore a trade-off exists between the production of cellulolytic and lignolytic enyzmes \cite{Sinsabaugh2010}. We found higher activities of both cellulolytic and lignolytic enzymes in N-rich litter, but their ratio was well correlated to lignin : carbohydrate decomposition. Lignin degradation yields less C and energy than carbohydrates degradation, but might provides additional N. Lignin decomposition was also suggested to allow decomposers direct competition by the early colonization of lignin rich sites (Treseder 2011). When C:X$_{substrate}$ : C:X$_{consumer}$ is low, as it was the case for lignin degrading litter in our experiment, additional carbon can not be used by decomposers to build up biomass. Therefore, the observed increase in lignolytic activities might result from a microbial strategy to optimize N allocation to cellulytic and lignolytic enzyme activities in a situation, where additional C can not be untilized by the decomposer due to a lack of nutrients. 

In summary, litter quality had profound influence on litter decomposition, including community decomposition and  lignin accumulation. Lignin decomposition within the first six months was highly variable between litter of different quality, and ranged from no lignin degradation to lignin degradation at bulk litter C loss rates. It was associated with nutrient-rich microbial biomass on low-nutrient litter, ie. a high difference between litter and decomposer stoichiomtry, low fungi:bacteria ratios and an elevated abundance of [gamma]-proteobacteria protein. Differences in lignin contents aquired within the first six months remained in place during later decomposition, although - regardless of litter quality - lignin was not further accumulated between 6 and 15 months.
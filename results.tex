\section*{Results}
\subsection*{Initial litter chemistry}
Initial litter chemistry of the four sites (Achenkirch, AK, Klausenleopoldsdorf, KL, Ossiach, OS, Schottenwald, SW), measured 14 days after incubation, is presented in supplemental table \ref{initstoech}. C:N ratios between 41:1 and 58:1 and C:P ratios between 700:1 and 1300:1 were found, N:P ratios ranged between 15:1 and 30:1. No significant changes occurred during litter incubation except a slight decrease of the C:N ratio (41.8:1 to 37.4:1) found in the most active litter type (SW) after 15 months. Fe concentrations were more than twice as high for OS (approx. 450 ppm) than for other litter types (approx. 200 ppm). Litter Mn also was highly variable between litter types, ranging between 170 and 2130 ppm. Changes of micro-nutrient concentrations during litter incubation were significant, but in all cases \textless 15\% of the initial concentration. In initial litter, lignin accounted for 28.9-31.2\% and carbohydrates for 25.9-29.2\% of the total peak area of all pyrolysis products.

\subsection*{Mass loss, respiration and soluble organic carbon}

Litter mass loss was not significant after 2 weeks and 3 months, and significant for 2 litter types after 6 months. After 15 months, litter mass loss was significant for all litter types, ranged between 5 and 12 \% of the initial dry mass, and was strongly correlated to litter N content (R=0.794, p\textless 0.001). Detailed results were reported by \cite{Mooshammer2011}.

Highest respiration rates were measured at the first measurement after 14 days incubation (150-350 \textmu g CO$_2$-C d$^{-1}$ g$^{-1}$ litter-C), which dropped to 75 to 100 \textmu g CO$_2$-C d$^{-1}$ g$^{-1}$ litter-C after 3 months. After 6 and 15 months, respiration rates for AK and OS further decreased, while SW and KL showed a second maximum in respiration after 6 months (fig \ref{fig:enz}). Accumulated respiration was correlated to litter mass loss (r=0.738, p\textless 0.001, n=20).

Soluble organic carbon concentrations decreased between the first three harvests (14 days to 6 months), and strongly increased to 15 months (from 0.1 to 0.7 mg C g$^{-1}$  d.w. to 1.5 to 4 mg C g$^{-1}$ d.w. after 15 months, fig. \ref{fig:enz}). After 14 days and 3 months, the highest soluble organic C concentration was found in SW litter followed by AK. Soluble organic C concentrations were weakly correlated with litter N content after 14 days (r=0.69, p\textless 0.001, n=20) and after 3 months (r = 0.65, p\textless 0.01, n=20), but were strictly correlated after 6 months (r=0.85, p\textless 0.001, n=20) and 15 months (r=0.90, p\textless 0.001, n=20).



\subsubsection*{Potential enzyme activities}
Within each timepoint, all potential extracellular enzyme activities were correlated with litter N and actual respiration rates(all R\textgreater 0.8, p\textless 0.001, n=20). Cellulase activity increased from first harvest onwards to 15 months, with a small depression after 6 months (Fig. \ref{fig:enz}), phenoloxidase and peroxidase activities reached their maximum after between 3 and 6 months (fig. \ref{fig:enz}). For all enzymes and at all time points, SW showed the highest and AK the lowest activity. Differences between these two sites were more pronounced in cellulase activity (SW 10x higher than AK) than in oxidative enzymes (4x higher). Conversely, the phenoloxidase/cellulase ratio was highest for AK and lowest for SW at all time points and decreased during litter decomposition. This indicates that microbial communities in AK litter invested more energy and nitrogen into degrading lignin and less into degrading carbohydrates than in litter from other sites. (fig. \ref{fig:enz}).

\subsubsection*{Microbial biomass abundance and community composition}
Microbial biomass contents ranged from 0.5 to 6 mg C g$^{-1}$ d.w., 0.05 to 0.55 mg N g$^{-1}$ d.w. and 0.05 to 0.35 mg P g$^{-1}$ litter d.w (fig. \ref{fig:mb}). After an initial increase in microbial biomass, in KL and OS microbial biomass remained constant after 3 months while AK and SW showed further microbial biomass growth reaching a maximum of microbial C and N contents after 6 months (AK also for P). Microbial C:N ratios ranged between 6:1 and 18:1, C:P ratios between 8:1 and 35:1, and N:P ratios between 0.5:1 and 3.5:1 (fig. \ref{fig:mb}).

Litter microbial biomass was stoichiometrically homeostatic during the first 6 months (no or negative correlations between microbial C:N:P and litter C:N:P, see also \cite{Mooshammer2011}), but after 15 months (microbial C:N:P ratios were significantly correlated to resource stoichiometry: R=0.53-0.64, all p\textless 0.002), when the homeostatic regulation coefficients \cite{Sterner2002} H\textsubscript{C:P}=1.68, H\textsubscript{C:N}=2.01, and H\textsubscript{N:P}=2.29 were found. Microbial C:N ratios after 3 and 6 months were within a tighly constrained range for each sampling time-point (14.5:1 to 18.2:1 and 6.9:1 to 9.0:1, respectively), but significantly different between the two sampling events. In contrast, microbial C:P and N:P ratios were less constrained, with the highest variance between litter from different sites after 3 months of incubation (fig. \ref{fig:mb}).

Metaproteome analysis  yielded between n and n different peptides per sample (one replicate for each litter type after 14 days, 3, 6, and 15 months), of which nn-nn\% could not be identified and were ignored in the further analysis. Of the identified peptides, only those assigned to bacteria or fungi were used for community profiling. Fungal proteins abundance was dominant in all litter types at all stages, but most prominent in SW and least pronounced in  AK. Fungi:bacteria (F:B) protein abundance ratios were highest after 14 days (5 to 12) and decreased during litter decomposition (1.7 to 3 after 15 months, see fig. \ref{fig:f2b}). Large initial differences in F:B ratios between litter from different sites decreased during decomposition. In addition, F:B ratios were measured on a DNA basis (qPCR) the results showing a similar pattern but with a much larger fungal DNA dominance (F:B ratios between 10-180). F:B ratios were highly correlated between protein- and log-transformed DNA-based estimates (r=0.785, p\textless 0.001, n=80).

Fungal communities were dominated by Ascomycetes, with smaller contributions by Basidiomycetes (<5\% of fungal protein).  Among the fungal classes found, Sordariomycetes and Eurotiomycetes were most abundant with further contributions of Dothideomycetes, Leothiomycetes and Saccharomycetes. Bacteria were dominated by Proteobacteria (mainly $\gamma$, declining, and $\alpha$- and $\beta$-Proteobacteria, increasing with litter decomposition) with minor contribution of Actinomycetes and Bacterioidetes (both increasing) and Thermotogae (decreasing, see supplemental figure [still missing]).

\subsection*{Pyrolysis-GC/MS and Lignin content}
In total 128 pyrolysis products were detected, quantified, identified and assigned to their substances of origin (suppl. tab. \ref{tab:phprod} -\ref{tab:nprod}). We found only minor changes in the relative concentration of litter pyrolysis products during decomposition, and differences between sites were small but well preserved during decomposition. However, the high precision and reproducibility of pyrolysis GC/MS analysis of litter allowed tracing small changes in lignin and carbohydrate abundance during decomposition. Lignin-derived compounds made up between 29 and 31 \% relative peak area (TIC) in initial litter, and increased by up to 3 \% over the first 6 months. Carbohydrate-derived pyrolysis products accounted for 26 to 29 \% in initial litter and decreased by up to 2.6 \% during litter decomposition. The initial (pyrolysis-based) lignin:carbohydrates index (LCI) were highly similar between litter from different collection sites, ranging between 0.517 and 0.533 (Fig. 4). During decomposition, the LCI increased by up to 9 \% of the initial value, the highest increase in with SW and a slight decrease in AK litter. All significant changes in LCI  occurred within the first 6 months, with insignificant changes thereafter (fig. \ref{fig:lci}). As differences in lignin and carbohydrate contents between 0-3 and 3-6 months were below significance, we analyzed differences for the two time intervals between 0-6 months and 6-15 months.

During the first 6 months, between one and 6 \% of the initial lignin pool and between 4 and 17\% of the initial carbohydrate pool were degraded (Fig. \ref{fig:degr}). Lignin decomposition was highest in AK and KL litter, while KL, OS and SW decomposed carbohydrates fastest. Lignin preference values (\% lignin decomposed : \%carbohydrates decomposed) were lowest in SW and highest in AK litter (Figure 5). In AK litter, lignin macromolecules were 50 \% more likely to be decomposed than carbohydrates, while in SW litter carbohydrates were 10 times more likely to be decomposed (fig. \ref{fig:degr}). Between 6 and 15 months, no further accumulation of lignin occurred, lignin and carbohydrates were both degraded at the same rates and their relative concentrations remained constant (fig. \ref{fig:degr}).

\subsection*{Correlations between lignin and carbohydrate decomposition and litter chemistry, microbial community and decomposition processes}

Relationships between lignin and carbohydrate degradation, litter chemistry, microbial biomass and decomposition processes were tested after 6 and 15 months (tables \ref{corrtable} and \ref{corrtable2}) including data presented by \cite{Mooshammer2011} and \cite{Leitner2011}. After 6 months, we found that the ratio of lignin/cellulose degradation was positively correlated with the ratio of phenoloxidase : cellulase (R=0.599, p=0.005, n=20) and peroxidase : cellulase (R=0.734 p\textless 0.001, n=20). Carbohydrate decomposition was positively correlated with litter N content, and negatively with litter C:N ratios and litter-microbial C:N imbalances. In contrast, lignin decomposition was negatively correlated to litter P, but positively with litter C:P and N:P ratios, and litter-microbial C:P and N:P imbalances (fig. \ref{fig:cor1}). After 15 months, the ratio of lignin : carbohydrate decomposition was no longer related to stoichiometry or elemental composition any more. Most interestingly, lignin and carbohydrate decomposition exhibited the same controls, being positively correlated to soluble organic C, litter N and litter P (table \ref{corrtable2}). Mass loss and accumulated respiration were positively correlated to lignin and carbohydrate decomposition (table \ref{corrtable2}), a pattern that we did not find for lignin decomposition in the early decomposition phase (table \ref{corrtable}). Protein abundance F:B ratios were negatively correlated to the ratios of lignin : cellulose decomposition and to LCI change during the first 6 months. In contrast, both lignin and carbohydrate decomposition rates were positively correlated with F:B ratios after 15 months, but not to the ratio of lignin : cellulose decomposition (fig. \ref{fig:f2b}).


To asses the interaction between litter chemistry, microbial community and degradation processes, we conducted a correspondence analysis (CA) on the relative protein abundances (fig. \ref{fig:metaprotpca}). The results indicate that incubations time (i.e. succession) is the dominant factor controlling the microbial community, with samples collected at the first (14 days) and the last (15 month) sampling event grouping closely together, while litter quality (i.e. different stoichiometry of litter collected at different sites) had a higher impact after 3 and 6 month. The first factor (CA 1), which explains 35.7 \% of the total variance, separates litter sampled after 15 months (positive values) from litter sampled earlier (negative values). Consequently, CA 1 is also correlated positively to incubation time and negatively to litter carbon content (i.e. falling C:N ratios during decomposition) . A number of bacterial taxa (Actinobacteria, Bacteroidetes, Alpha- and Betaproteobacteria), and two fungal classes (Leotiomycetes and Tremellomycetes) are positively correlated to this factor i.e. have increased abundance after 15 months, while Cyanobacteria, Epsilonproteobacteria and Saccharomycetes are negatively correlated. CA 2, which explains further 26.0 \% variance, separates litter sampled within the first 6 months. Dothideomycetes and Sordariomycetes are positively and Gammaproteobacteria negatively correlated to this factor, which also correlates to the F:B protein abundance ratio. Litter collected 14 days after inoculation have the highest scores on CA 2, while sites with active lignin degradation within the first 6 months (AK, KL) have the most negative scores. The axis was furthermore correlated to the microbial biomass P content and C:P and N:P imbalances (and free NH\textsubscript{4}\textsuperscript{+}, not shown). For samples analyzed after 6 months, where direct comparison to lignin degradation rates is possible, significant correlations to CA 2 were found for lignin : carbohydrate degradation (r=-0.97, p=0.028), \% Lignin loss : \% Carbon loss (r=-0.96, p=0.040) and  LCI increase (r=0.973, p=0.027), even though of directly comparable samples was very low (n=4). Differences in CA2 are strongly decreased after 15 months, suggesting that the differences in the microbial community found within the first 6 months are lost with ongoing succession of the decomposer community. Litter N and P contents were not correlated to either factor, although differences in resource quality evidently effected community composition after 3 and 6 months, as can be seen in the differences in the developing communities as observed in CA 2. Correlation of CA factors with litter stoichiometry, and microbial stoichiometry, and the abundance of the analyzed taxa are provided in supplemental table \ref{catab}.

%Linear vector fittings revealed a significant correlations for incubation time and litter C content, C:P and N:P imbalances and the F:B protein abundance ratio. No significant fitting could be found for litter N and P contents. 

\section*{Results}
\subsection*{Initial litter chemistry}
Initital litter chemistry of the four sites (Achenkirch, AK, Klausenleopoldsdorf, KL, Ossiach, OS, Schottenwald, SW), measured 14 days after incubation, is presented in table \ref{initstoech}. C:N ratios between 41:1 and 58:1 and C:P ratios between 700:1 and 1300:1 were found, N:P ratios ranged between 15:1 and 30:1. No significant changes occurred during litter incubation except a slight decrease of the C:N ratio (41.8:1 to 37.4:1) found in the most active litter type (SW) after 15 month. Fe concentrations were more than twice as high for OS (approx. 450 ppm) than for other litter types (approx. 200 ppm). Litter Mn also was highly variable between litter types, ranging between 170 and 2130 ppm. Changes of micro-nutrient concentrations during litter incubation were significant, but in all cases \textless 15\% of the initial concentration. In initial litter, lignin accounted for 28.9-31.2\% and carbohydrates for 25.9-29.2\% of the total peak area of all pyrolysis products.

\subsection*{Mass loss, respiration and soluble organic carbon}

Litter mass loss was not significant after 2 weeks and 3 months, and significant for 2 litter types after 6 months. After 15 months, litter mass loss was significant for all litter types, ranged between 5 and 12 \% of the initial dry mass, and was strongly correlated to litter N content (R=0.794, p\textless 0.001). Detailed results were reported by \cite{Mooshammer2011}.

Highest respiration rates were measured at the first measurement after 14 days incubation (150-350 \textmu g CO$_2$-C d$^{-1}$ g$^{-1}$ litter-C), which dropped to 75 to 100 \textmu g CO$_2$-C d$^{-1}$ g$^{-1}$ litter-C after 3 months. After 6 and 15 months, respiration rates for AK and OS further decreased, while SW and KL showed a second maximum in respiration after 6 months (fig \ref{fig:enz}). Accumulated respiration was correlated to litter mass loss (r=0.738, p\textless 0.001, n=20).%\footnote{or r=0.961, p=0.038, n=4 if means are compared are used.}

Soluble organic carbon concentrations decreased between the first three harvests (14 days to 6 months), and strongly increased to 15 months (from 0.1 to 0.7 mg C g$^{-1}$  d.w. to 1.5 to 4 mg C g$^{-1}$ d.w. after 15 months, fig. \ref{fig:enz}). After 14 days and 3 months, the highest soluble organic C concentration was found in SW litter followed by AK. Soluble organic C concentrations were weakly correlated with litter N content after 14 days (r=0.69, p\textless 0.001) and after 3 months (r = 0.65, p\textless 0.01), but were strictly correlated after 6 months (r=0.85, p\textless 0.001) and 15 months (r=0.90, p\textless 0.001).

%\begin{figure*}[t]
%\vspace*{2mm}
%\begin{center}
%\setkeys{Gin}{width=0.5\textwidth}
%<<doc, echo=F, results=hide, fig=T, height=7, width=7>>=
%timeseries(alldata$DOC/1000, alldata$days, alldata$Litter, pch=pch, pt.bg=colscale,  xlab="Incubation (days)", ylab="mg C g-1 d.w.", main="Extractable carbon", lwd=2)
%legend("bottomright", pt.bg=colscale, pch=pch, typlev, lty=1, lwd=2)
%@
%\end{center}
%\caption{Extractable organic carbon. Error bars indicate standard errors (n=5).}
%\label{fig:doc}
%\end{figure*}

\subsubsection*{Potential enzyme activities}
Potential extracellular enzyme activities were correlated with litter N, respiration and other decomposition processes (all R\textgreater 0.8, p\textless 0.001). Cellulase activity increased from first harvest onwards to 15 months, with a small depression after 6 months (Fig. \ref{fig:enz}), phenoloxidase and peroxidase activities reached their maximum after between 3 and 6 months (fig. \ref{fig:enz}). For all enzymes and at all time points, SW showed the highest and AK the lowest activity. Differences between these two sites were more pronounced in cellulase activity (SW 10x higher than AK) than in oxidative enzymes (4x higher). Conversely, the phenoloxidase/cellulase ratio was highest for AK and lowest for SW at all time points and decreased during litter decomposition. This indicates that microbial communities in AK litter invested more energy and nitrogen into degrading lignin and less into degrading carbohydrates than in litter from other sites. (fig. \ref{fig:enz}).

\subsubsection*{Microbial biomass abundance and community composition}
Microbial biomass contents ranged from 0.5 to 6 mg C g$^{-1}$ d.w., 0.05 to 0.55 mg N g$^{-1}$ d.w. and 0.05 to 0.35 mg P g$^{-1}$ litter d.w (fig. \ref{fig:mb}). In KL and OS microbial biomass buildup reached a plateau after 3 months, AK and SW showed further microbial biomass growth reaching a maximum of microbial C and N contents after 6 months (AK also for P). Microbial C:N ratios ranged between 6 and 18, C:P ratios between 8 and 35, and N:P ratios between 0.5 and 3.5 (fig. \ref{fig:mb}).

Litter microbial biomass was stoichiometrically homeostatic during the first 6 months (no or negative correlations between microbial C:N:P and litter C:N:P, see also \cite{Mooshammer2011}), but after 15 months (microbial C:N:P ratios were significantly correlated to resource stoichiometry: R=0.53-0.64, all p\textless 0.002), when the homeostatic regulation coefficients \cite{Sterner2002} H\textsubscript{C:P}=1.68, H\textsubscript{C:N}=2.01, and H\textsubscript{N:P}=2.29 were found. Microbial C:N ratios after 3 and 6 month were within a tighly constrained ranges (14.5:1 to 18.2:1 and 6.9:1 to 9.0:1, respectively), but significantly different between the two time points. Microbial C:P and N:P ratios were less constrained, with the highest variance between litter from different sites after 3 months of incubation (fig. \ref{fig:mb}).

Fungi:bacteria (F:B) ratios derived from metaproteomics data of the litter (one replicate per litter type and harvest) were highest after 14 days (5 to 12) and decreased during litter decomposition (1.7 to 3 after 15 months, see fig. \ref{fig:f2b}) . The large differences in fungi:bacteria ratios between litter types decreased during decomposition. Fungal proteins abundance was dominant in all litter types at all stages, but most prominent in SW and least pronounced in  AK. The fungi:bacteria ratios were negatively correlated to the ratios of lignin:cellulose decomposition and to LCI change during the first 6 months. In contrast, lignin decomposition rates were positively correlated with fungi:bacteria ratios after 15 months but not to the ratios of lignin:cellulose decomposition (fig. \ref{fig:f2b}). In addition, fungi: bacteria ratios were measured on a DNA basis (qPCR) the results showing a similar pattern between litter types and harvests but with a much larger fungal DNA dominance (ratios between 10-180). Fungi:bacteria ratios were highly correlated between protein- and DNA-based estimates (r=0.801, p\textless 0.001, with log-transformed qPCR ratios).

\subsection*{Pyrolysis-GC/MS and Lignin content}
In total 128 pyrolysis products were detected, quantified, identified and assigned to their origin (suppl. tab. \ref{tab:phprod} -\ref{tab:nprod}). We found only minor changes in the relative concentration of litter pyrolysis products during decomposition, and differences between sites were small but well preserved during decomposition. However, the high precision and reproducibility of pyrolysis GC/MS analysis of litter allowed tracing small changes in lignin and carbohydrate abundance during decomposition. Lignin-derived compounds made up between 29 and 31 \% relative peak area (TIC) in initial litter, and increased by up to 3 \% over the first 6 months. Carbohydrate-derived pyrolysis products accounted for 26 to 29 \% in initial litter and decreased by up to 2.6 \% during litter decomposition. The pyrolysis-based LCI index  showed a small range between 0.517 and 0.533 initially (Fig. 4). During decomposition, LCI increased by up to 9 \% of the initial value, with SW showing the highest increase while in AK litter the LCI slightly decreased. The changes in LCI almost completely occurred over the first 6 months, with insignificant changes thereafter (fig. \ref{fig:lci}).

During the first 6 months of litter decomposition, between one and 6 \% of the initial lignin pool and between 4 and 17\% of the initial carbohydrate pool were degraded (Fig. \ref{fig:degr}). Lignin decomposition was highest in AK and KL litter, while KL, OS and SW decomposed carbohydrates fastest. Lignin preference values (\% lignin decomposed/\%carbohydrates decomposed) were lowest in SW and highest in AK litter (Figure 5). In AK litter, lignin macromolecules were 50 \% more likely to be decomposed than carbohydrates, while in SW litter carbohydrates were 10 times more likely to be decomposed (fig. \ref{fig:degr}). Between 6 and 15 months, no further accumulation of lignin occurred, lignin and carbohydrates were both degraded at the same rates and their relative concentrations remained constant (fig. \ref{fig:degr}).

\subsection*{Correlations between lignin and carbohydrate decomposition and litter chemistry, microbial community and decomposition processes}

Relationships between lignin and carbohydrate degradation, litter chemistry, microbial biomass and decomposition processes were tested after 6 and 15 months (tables \ref{corrtable} and \ref{corrtable2}) including data presented by \cite{Mooshammer2011} and \cite{Leitner2011}. After 6 months, we found that the ratio of lignin/cellulose degradation was positively correlated with the ratio of phenoloxidase/cellulase (R=0.599, p=0.005) and peroxidase/cellulase (R=0.734 p\textless 0.001, table \ref{corrtable}). Carbohydrate decomposition was positively correlated with litter N content, and negatively with litter C:N ratios and litter-microbial C:N imbalances. In contrast, lignin decomposition was negatively correlated to litter P, but positively with litter C:P and N:P ratios, and litter-microbial C:P and N:P imbalances (fig. \ref{fig:cor1}). After 15 months, the ratio of lignin/carbohydrate decomposition was not related to stoichiometry or elemental composition any more. Most interestingly, lignin and carbohydrate decomposition exhibited the same controls, being positively correlated to soluble organic C, litter N and litter P (table \ref{corrtable2}). Mass loss and accumulated respiration were positively correlated to lignin and carbohydrate decomposition (table \ref{corrtable2}), a pattern that we did not find for lignin decomposition in the early decomposition phase (table \ref{corrtable}).
